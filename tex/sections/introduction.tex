Finding smooth approximations to points distributed in the unit sphere play an important role in directional statistics. 
An important application of such problem is the estimation of Apparent Polar Wander Paths (APWP) in paleomagnetism which describe the relative continental drift of tectonic plates.
Here, remanent magnetization of rocks is used to estimate directions of the ancient Earth magnetic field. 
The goal here is to model directions of vectors and ignore the magnitude of the same, reason why working in the unit sphere is the natural setup. 

In recent years, there has been an increasing interest in combining machine learning algorithms with differential equations. 
Solutions of differential equations can be used to represent a broad class of functions that can be use for regression purposes. 
This is the case with Universal Differential Equations (UDE) or neural differential equations, where observations are modelled as the numerical solution of a differential equation that has embedded some form of regressor (for example, a neural network) inside the differential equation \cite{rackauckas2020universal}.  

In this work, we are interested in modelling directional data in three-dimensional spaces using UDEs. 
Our model addresses the following three major points for data supported in the sphere. 
We further beleive these consideration are a first step toward more complex models that combine statistical regression with differential equations. 
\begin{enumerate}
    \item \textbf{Constrained observations.} In this work we are interested in model data supported in the unit sphere. More generally, there is a rich literature of differential equations with associated conservation laws that will play a similar role. 
    \item \textbf{Path regularization.} We will start considering the general case on non-parametric regression with no regularization, but we are going to consider different types of regularization that are desired at the moment of fitting APWPs.
    \item \textbf{Temporal uncertainties.} Ideally, we want a model that can deal with both uncertainties in space and time. Here we want to emphasize the importance of accounting for uncertainties in time, since paleopole dating is a difficult tasks and time estimates tend to carry important uncertainties. 
\end{enumerate}
We hope the contributions of this work can help both paleomagnetism to fit APWPs but also the broader community of researchers applying neural differential equations to physical systems. 


\subsection{Setup}

Consider a sequence of pairs $(t_i, y_i)$, $i=1, 2, \ldots, N$, where each $t_i$ corresponds to an observed time and each $y_i \in S^2$ is an unit vector supported the the unit sphere $S^2 = \{ x \in \R^3 : \| x \|_2 = 1 \}$. 
% Each $x$ can be represented in Cartesian coordinates $\mathbf x= (x, y, z)$, with $\| \mathbf x \|_2 = 1$, or in spherical coordinates with its latitude and co-longitude $(\theta, \phi)$, with 
% \begin{equation}
%     x = \cos \theta \sin \phi, \, y = \sin \theta \sin \phi, \, z = \cos \phi. 
% \end{equation}
% In order to avoid confusion, we are going to use bold characters just for vectors. 
A rotation matrix $R(p, \theta) \in \R^{3 \times 3}$ is defined by an axis of rotation $p \in S^2$ called Euler pole and an angle of rotation $\theta$.
Given some vector $x_0$, the rotation matrix $R(p, \theta)$ transforms $x_0$ into $R(p, \theta) x_0$.
The space of rotations in three dimensions $SO(3)$ is a Lie group with an associated Lie algebra associated to the infinitesimal generators of rotations. 
A rotation matrix can be approximated at first-order with respect to the rotation angle as
\begin{equation}
    R(p, \theta) 
    = 
    I + \theta \, \text{skewt}(d) + \mathcal O (\theta^2 ),
\end{equation}
where $ \text{skewt}(p)$ represents the skew-symmetric matrix defined by the rotation axis $p$, which satisfies $\text{skewt}(p)  x =p \times x$ for all vector $x \in \R^3$. 
If we call $\theta = \omega \Delta t$, with now $\omega$ some angular velocity and $\Delta t$ some time interval over we apply the rotation, as $\Delta t \rightarrow 0$, the successive application of rotation matrix with same Euler pole $p$ correspond to solutions of the differential equation 
\begin{equation}
    \frac{dx}{dt} (t)
    = 
    L(t) \times x(t) %, \quad \mathbf x(0) = \mathbf x_0
    \label{eq:sphere-ode}
\end{equation}
where $L = \omega p$ can be associated to the angular momentum of the solid sphere, that is, a vector aligned with the rotation axis and with norm equals to $\omega$. 

We will assume that each $y_i$ is a sample from the von Mises-Fisher distribution \cite{fisher1953dispersion, Watson_1982} in the sphere with mean direction $x(t_i)$ and concentration parameter $\kappa_i$, 
\begin{equation}
    p(y_i | x, \kappa_i)
    = 
    \frac{\kappa_i}{4 \pi \sinh (\kappa_i)} e^{- \kappa_i x(t_i)^T y_i},
\end{equation}
where $x(t)$ is the solution of a differential equation. 



\subsection{Related work}

Running mean

A first approach to fitting smooth path to spherical data was introduced in \cite{Jupp-1987-spherical}, where the authors consider a loss function of the form 
\begin{equation}
    \sum_{i=1}^n \cos^{-1}(y_i^T f(t_i))
    + 
    \lambda
    \int \| \nabla_\text{cov}^2 f(t) \|_2^2 dt,
    \label{eq:jupp_splines_sphere}
\end{equation}
where the minimization is perform over all functions $f: \R \mapsto S^2$ and $\nablacov^2 f  = (I - ff^T) f''$ is the second covariant derivative. In the same paper, the authors derive an methodology to minimize such loss function by introducing an \textit{unrolling} procedure that maps points and paths from the sphere to the two-dimensional plane. This leads to the same loss function used in cubic splines but now in Euclidean space, where covariant derivative match the usual derivative. 